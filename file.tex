\documentclass[13pt, a4paper, twoside]{article}
\usepackage{setspace}
\setstretch{1.5}
\usepackage[utf8]{inputenc}
\usepackage[russian]{babel}
\usepackage{amsmath}
\usepackage{graphicx}
\graphicspath{ {/home/vlad/chm-zadan/chm/} }
\usepackage{float} 
\usepackage{booktabs}
\usepackage{tabto} 
\usepackage{xcolor}
\usepackage{hyperref}
\definecolor{urlcolor}{HTML}{00008b}
\hypersetup{pdfstartview=FitH,  linkcolor=linkcolor,urlcolor=urlcolor, colorlinks=true}


\title{Численные методы. Практическое задание.}
\author{Швецов Владислав Андреевич. 213 группа.}
\date{\today}

\begin{document}

\maketitle


\tab
\tab

\section{Задача 2.3}
\tab Вычеслить значения функции $f(x) = \int_{0}^{x} g(t) \,dt$ для $x$ в диапазоне от $-2$ до $2$ с шагом $0.5$, используя численное интегрирование:

\begin{enumerate}
    \item $g(t) = sinh\ 2t \cdot (x^2 + 1)$
    \item $g(t) = -\frac{5t\cdot cos\ t}{(1\ +\ 2t^2)^3} $
\end{enumerate}

Построить интерполяционный полином Лагранжа, используя вычесленные значения.
Сравнить приближенные значения фунеции $f(x)$ с полученными значениями на сетке с шагом $0.1$ в том же диапозоне. Оценить абсолютную и относительную ошибку.
Подобрать более эфективный метод приближения функции $f(x).$

\section{Teoрия}
\subsection{Интерполяционный полином Лагранжа}

\tub Интерполяционный полином Лагранжа \-- это полином, использующийсяя для аппроксимации функции по её значениям в заданных точках.

Построение. Пусть есть набор узлов \( (x_0, y_0), (x_1, y_1), \ldots, (x_n, y_n) \), где \( x_i \) \-- различные точки на интервале, а \( y_i \)\-- значения функции в этих точках.
Интерполяционный полином Лагранжа для таких узлов задается следующей формулой:

\[
P_n(x) = y_0 L_0(x) + y_1 L_1(x) + \ldots + y_n L_n(x),
\]где \[L_i(x) = \prod_{j=0, j\neq i}^{n} \frac{x - x_j}{x_i - x_j}\]
\tabПолином Лагранжа имеет следующие свойства:
\begin{enumerate}
    \item Интерполяционный многочлен Лагранжа имеет степень не выше \( n \), где \( n \) - количество узлов.
    \item Значения многочлена в узлах совпадают с заданными значениями функции.
    \item Существует единственный многочлен Лагранжа для данного набора узлов.
\end{enumerate}

\subsection{Узлы Чебышева}
\tab Узлы Чебышёва - это особые узлы интерполяции, используемые для уменьшения эффекта Рунге при построении интерполяционных полиномов.
Для интервала $[-1, 1]$ для натурального $n$ узлы Чебышева задаются по формуле:
\[
x_i = \cos\left(\frac{(2i - 1)\pi}{2n}\right),
\]
На интервале $[a, b]$ узлы Чебышева $x_i$ находятся при помощи афинного преобразования отрезков:
\[
x_i = \frac{a + b}{2} + \frac{b - a}{2} \cos\left(\frac{(2i - 1)\pi}{2n}\right),
\]
где $i = 1, 2 \ldots, n$ - номер узла, $n$ - количество узлов на интервале $[a, b]$.

Узлы Чебышевы обладают следующими свойствами:
\begin{enumerate}
    \item Интерполяция по узлам Чебышёва наименее подвержена эффекту Рунге, из-за того, что узлы больше распределяются по краям отрезка.
    \item Узлы Чебышева обеспечивают оптимальную сходимость интерполяционных полиномов к интерполируемой функции.
\end{enumerate}
Узлы Чебышёва могут улучшить точность интерполяции, особенно для функций, сильно изменяющихся на краях интерполяционного интервала.
\subsection{Применение интерполяции в реальной жизни}
\textbf{Компьютерное зрение:}

В области обработки изображений и компьютерного зрения интерполяция применяется для повышения разрешения изображений или восстановления отсутствующих частей изображения.


\textbf{Визуализация данных:}

Интерполяция часто используется для создания плавных кривых на графиках, что позволяет эффективно визуализировать данные. Это особенно важно в графике и компьютерной графике.

\textbf{Анализ данных во времени:}

В таких областях, как финансовый анализ или моделирование климата, где данные представлены в виде временных рядов, интерполяция используется для увеличения частоты наблюдений или для восполнения пропущенных значений.


   
\textbf{Аудио- и видеопроизводство:}

В обработке аудио и видео интерполяция используется для улучшения качества звука и изображения, а также для изменения частоты кадров в видео.

   
\textbf{Финансовое моделирование:}

В области финансов интерполяция помогает аппроксимировать такие показатели, как процентные ставки и кривые доходности, что улучшает точность финансовых моделей.
\section{Применимаость}
\subsection{Построение интерполяционного полинома Лагранжа}
\tub Нахождение значений функции $f_1(x)$(см. рис 1), $f_2(x)$(см. рис 2) и последующего использования их для построения полинома Лагранжа, с помощью интеграла Симпсона. Вывод программы:
\begin{verbatim}
                    ________Function 1_________
                    x:  -2.000000 y:  65.952515
                    x:  -1.500000 y:  14.767509
                    x:  -1.000000 y:   2.767032
                    x:  -0.500000 y:   0.339915
                    x:   0.000000 y:   0.000000
                    x:   0.500000 y:   0.339915
                    x:   1.000000 y:   2.767032
                    x:   1.500000 y:  14.767509
                    x:   2.000000 y:  65.952515
                    
                    Function + Chebyshov nodes
                    x:   1.969616 y:  60.423157
                    x:   1.732051 y:  30.052754
                    x:   1.285575 y:   7.414043
                    x:   0.684040 y:   0.802122
                    x:   0.000000 y:   0.000000
                    x:  -0.684040 y:   0.802122
                    x:  -1.285575 y:   7.414043
                    x:  -1.732051 y:  30.052754
                    x:  -1.969616 y:  60.423157
\end{verbatim}
\begin{center}
  рис. 1
\end{center}

\begin{verbatim}
                    ________Function 2_________
                    x:  -2.000000 y:  -0.504968
                    x:  -1.500000 y:  -0.506644
                    x:  -1.000000 y:  -0.488853
                    x:  -0.500000 y:  -0.330292
                    x:   0.000000 y:   0.000000
                    x:   0.500000 y:  -0.330292
                    x:   1.000000 y:  -0.488853
                    x:   1.500000 y:  -0.506644
                    x:   2.000000 y:  -0.504968
                    
                    Function + Chebyshov nodes
                    x:   1.969616 y:  -0.505141
                    x:   1.732051 y:  -0.506367
                    x:   1.285575 y:  -0.504220
                    x:   0.684040 y:  -0.422646
                    x:   0.000000 y:  -0.000000
                    x:  -0.684040 y:  -0.422646
                    x:  -1.285575 y:  -0.504220
                    x:  -1.732051 y:  -0.506367
                    x:  -1.969616 y:  -0.505141
\end{verbatim}
\begin{center}
  рис. 2
\end{center}

\subsection{Применимость интерполяционного полинома Лагранжа}
\tub Вывод программы значений функции, интерполяционного полинома Лагранжа, и сравнение их для функции $f_1(x)$ (см. таблица. 1) и $f_2(x)$ (см. таблица 2).







\subsection{Применимость интерполяционного полинома Лагранжа c узлами Чебышева}
\tubВывод программы значений функции, интерполяционного полинома Лагранжа с использование узлов Чебышева и сравнение их, для функции $f_1(x)$ (см. таблица. 3) и $f_2(x)$ (см. таблица 4).





\section{Реализация на языке C}
\tub Исхдный код программы, реализованной на языке C, можно посмотреть в файле \textbf{zadan.c}

\section{Анализ и вывод}
\tubЧисленное интегрирование функции $f(x)$ с использованием интерала Симпсона позволяет найти ее значения с высокой точностью, и интерполяция с помощью полинома Лагранжа дает хорошие приближения для большинства значений $x$.
Однако для функций с большими колебаниями или сильными изменениями на краях интервала использование узлов Чебышева значительно улучшает точность интерполяции.
Метод Лагранжа является хорошим выбором для малых интервалов и функций с достаточно гладкими изменениями, но при необходимости высокой точности на большом интервале следует рассмотреть альтернативные методы интерполяции, такие как сплайны или использование узлов Чебышева.

\section{Литература}
\begin{enumerate}
    \item \href{https://ftp.vtyulb.ru/Костомаров.%20Вводные%20лекции%20по%20ЧМ.pdf}{Д.П. Костомаров, А.П. Фаворский.(2004)'Вводные лекции по численным методам',Логос.}

    \item \href{https://teach-in.ru/file/synopsis/pdf/numerical-methods-part-1-M.pdf}{Г.М. Кобельков. Численные методы.Часть 1.(лекции).}
    \item \href{https://en.wikipedia.org/wiki/Lagrange_polynomial}{Полинм Лагража. Статья.}
    \item \href{https://en.wikipedia.org/wiki/Chebyshev_nodes}{Узлы Чебышева. Статья.}

\section{Таблицы}
\end{enumerate}
\begin{table}[!ht]
    \centering
    \begin{tabular}{|l|l|l|l|l|}
    \hline
        x & f (x) & P (x) & Абсолютная ошибка & Относительная ошибка \\ \hline
        -2.0 & 65.952515 & 65.952515 & 0.000000 & 1.000000 \\ \hline
        -1.9 & 49.369345 & 49.459309 & 0.000001 & 1.001822 \\ \hline
        -1.8 & 36.796128 & 36.882257 & 0.000002 & 1.002341 \\ \hline
        -1.7 & 27.293520 & 27.346775 & 0.000003 & 1.001951 \\ \hline
        -1.6 & 20.136732 & 20.157849 & 0.021117 & 1.001049 \\ \hline
        -1.5 & 14.767509 & 14.767509 & 0.000000 & 1.000000 \\ \hline
        -1.4 & 10.756670 & 10.747147 & 0.009523 & 0.999115 \\ \hline
        -1.3 & 7.774920 & 7.764173 & 0.010747 & 0.998618 \\ \hline
        -1.2 & 5.570146 & 5.562488 & 0.007657 & 0.998625 \\ \hline
        -1.1 & 3.949762 & 3.946328 & 0.003434 & 0.999130 \\ \hline
        -1.0 & 2.767032 & 2.767032 & 0.000000 & 1.000000 \\ \hline
        -0.9 & 1.910458 & 1.912349 & 0.001890 & 1.000989 \\ \hline
        -0.8 & 1.295599 & 1.297904 & 0.002305 & 1.001779 \\ \hline
        -0.7 & 0.858743 & 0.860490 & 0.001746 & 1.002034 \\ \hline
        -0.6 & 0.552067 & 0.552884 & 0.000818 & 1.001481 \\ \hline
        -0.5 & 0.339915 & 0.339915 & 0.000000 & 1.000000 \\ \hline
        -0.4 & 0.195987 & 0.195535 & 0.000452 & 0.997696 \\ \hline
        -0.3 & 0.101217 & 0.100704 & 0.000513 & 0.994928 \\ \hline
        -0.2 & 0.042215 & 0.041888 & 0.000327 & 0.992260 \\ \hline
        -0.1 & 0.010147 & 0.010049 & 0.000098 & 0.990323 \\ \hline
        0.0 & 0.000000 & 0.000000 & 0.000000 & 0.241644 \\ \hline
        0.1 & 0.010147 & 0.010049 & 0.000098 & 0.990323 \\ \hline
        0.2 & 0.042215 & 0.041888 & 0.000327 & 0.992260 \\ \hline
        0.3 & 0.101217 & 0.100704 & 0.000513 & 0.994928 \\ \hline
        0.4 & 0.195987 & 0.195535 & 0.000452 & 0.997696 \\ \hline
        0.5 & 0.339915 & 0.339915 & 0.000000 & 1.000000 \\ \hline
        0.6 & 0.552067 & 0.552884 & 0.000818 & 1.001481 \\ \hline
        0.7 & 0.858743 & 0.860490 & 0.001746 & 1.002034 \\ \hline
        0.8 & 1.295599 & 1.297904 & 0.002305 & 1.001779 \\ \hline
        0.9 & 1.910458 & 1.912349 & 0.001890 & 1.000989 \\ \hline
        1.0 & 2.767032 & 2.767032 & 0.000000 & 1.000000 \\ \hline
        1.1 & 3.949762 & 3.946328 & 0.003434 & 0.999130 \\ \hline
        1.2 & 5.570146 & 5.562488 & 0.007657 & 0.998625 \\ \hline
        1.3 & 7.774920 & 7.764173 & 0.010747 & 0.998618 \\ \hline
        1.4 & 10.756670 & 10.747147 & 0.009523 & 0.999115 \\ \hline
        1.5 & 14.767509 & 14.767509 & 0.000000 & 1.000000 \\ \hline
        1.6 & 20.136732 & 20.157849 & 0.021117 & 1.001049 \\ \hline
        1.7 & 27.293520 & 27.346775 & 0.053255 & 1.001951 \\ \hline
        1.8 & 36.796128 & 36.882257 & 0.086129 & 1.002341 \\ \hline
        1.9 & 49.369345 & 49.459309 & 0.089965 & 1.001822 \\ \hline
    \end{tabular}
    \caption{Вывод программы, представленный в табличном представлении для $f_1(x)$}
\end{table}


\begin{table}[!ht]
    \centering
    \begin{tabular}{|l|l|l|l|l|}
    \hline
        x & f (x) & P (x) & Абсолютная ошибка & Относительная ошибка \\ \hline
        -2.0 & -0.504968 & -0.504968 & 0.000000 & 1.000000 \\ \hline
        -1.9 & -0.505534 & -0.967091 & 0.461557 & 1.913010 \\ \hline
        -1.8 & -0.506060 & -1.002221 & 0.496161 & 1.980439 \\ \hline
        -1.7 & -0.506490 & -0.852283 & 0.345794 & 1.682726 \\ \hline
        -1.6 & -0.506731 & -0.661943 & 0.155212 & 1.306302 \\ \hline
        -1.5 & -0.506644 & -0.506644 & 0.000000 & 1.000000 \\ \hline
        -1.4 & -0.506017 & -0.415033 & 0.090984 & 0.820196 \\ \hline
        -1.3 & -0.504531 & -0.386498 & 0.118033 & 0.766053 \\ \hline
        -1.2 & -0.501706 & -0.404461 & 0.097245 & 0.806171 \\ \hline
        -1.1 & -0.496829 & -0.446071 & 0.050758 & 0.897836 \\ \hline
        -1.0 & -0.488853 & -0.488853 & 0.000000 & 1.000000 \\ \hline
        -0.9 & -0.476274 & -0.514861 & 0.038587 & 1.081020 \\ \hline
        -0.8 & -0.457001 & -0.512820 & 0.055819 & 1.122142 \\ \hline
        -0.7 & -0.428307 & -0.478711 & 0.050404 & 1.117682 \\ \hline
        -0.6 & -0.387018 & -0.415202 & 0.028184 & 1.072822 \\ \hline
        -0.5 & -0.330292 & -0.330292 & 0.000000 & 1.000000 \\ \hline
        -0.4 & -0.257406 & -0.235486 & 0.021919 & 0.914845 \\ \hline
        -0.3 & -0.172691 & -0.143778 & 0.028912 & 0.832577 \\ \hline
        -0.2 & -0.088411 & -0.067684 & 0.020728 & 0.765555 \\ \hline
        -0.1 & -0.024241 & -0.017508 & 0.006733 & 0.722242 \\ \hline
        0.0 & -0.000000 & -0.000000 & 0.000000 & 0.650550 \\ \hline
        0.1 & -0.024241 & -0.017508 & 0.006733 & 0.722242 \\ \hline
        0.2 & -0.088411 & -0.067684 & 0.020728 & 0.765555 \\ \hline
        0.3 & -0.172691 & -0.143778 & 0.028912 & 0.832577 \\ \hline
        0.4 & -0.257406 & -0.235486 & 0.021919 & 0.914845 \\ \hline
        0.5 & -0.330292 & -0.330292 & 0.000000 & 1.000000 \\ \hline
        0.6 & -0.387018 & -0.415202 & 0.028184 & 1.072822 \\ \hline
        0.7 & -0.428307 & -0.478711 & 0.050404 & 1.117682 \\ \hline
        0.8 & -0.457001 & -0.512820 & 0.055819 & 1.122142 \\ \hline
        0.9 & -0.476274 & -0.514861 & 0.038587 & 1.081020 \\ \hline
        1.0 & -0.488853 & -0.488853 & 0.000000 & 1.000000 \\ \hline
        1.1 & -0.496829 & -0.446071 & 0.050758 & 0.897836 \\ \hline
        1.2 & -0.501706 & -0.404461 & 0.097245 & 0.806171 \\ \hline
        1.3 & -0.504531 & -0.386498 & 0.118033 & 0.766053 \\ \hline
        1.4 & -0.506017 & -0.415033 & 0.090984 & 0.820196 \\ \hline
        1.5 & -0.506644 & -0.506644 & 0.000000 & 1.000000 \\ \hline
        1.6 & -0.506731 & -0.661943 & 0.155212 & 1.306302 \\ \hline
        1.7 & -0.506490 & -0.852283 & 0.345794 & 1.682726 \\ \hline
        1.8 & -0.506060 & -1.002221 & 0.496161 & 1.980439 \\ \hline
        1.9 & -0.505534 & -0.967091 & 0.461557 & 1.913010 \\ \hline
    \end{tabular}
    \caption{Вывод программы, представленный в табличном представлении для $f_2(x)$}
\end{table}

\begin{table}[!ht]
    \centering
    \begin{tabular}{|l|l|l|l|l|}
    \hline
        x & f (x) & P (x) + Узлы Чебышева & Абсолютная ошибка & Относительная ошибка \\ \hline
        -2.0 & 65.952515 & 65.929406 & 0.023108 & 0.999650 \\ \hline
        -1.9 & 49.369345 & 49.390083 & 0.020738 & 1.000420 \\ \hline
        -1.8 & 36.796128 & 36.808375 & 0.012247 & 1.000333 \\ \hline
        -1.7 & 27.293520 & 27.288318 & 0.005202 & 0.999809 \\ \hline
        -1.6 & 20.136732 & 20.121373 & 0.015359 & 0.999237 \\ \hline
        -1.5 & 14.767509 & 14.751773 & 0.015736 & 0.998934 \\ \hline
        -1.4 & 10.756670 & 10.747136 & 0.009533 & 0.999114 \\ \hline
        -1.3 & 7.774920 & 7.773759 & 0.001161 & 0.999851 \\ \hline
        -1.2 & 5.570146 & 5.576042 & 0.005897 & 1.001059 \\ \hline
        -1.1 & 3.949762 & 3.959557 & 0.009795 & 1.002480 \\ \hline
        -1.0 & 2.767032 & 2.777260 & 0.010228 & 1.003696 \\ \hline
        -0.9 & 1.910458 & 1.918426 & 0.007967 & 1.004170 \\ \hline
        -0.8 & 1.295599 & 1.299898 & 0.004299 & 1.003318 \\ \hline
        -0.7 & 0.858743 & 0.859280 & 0.000537 & 1.000625 \\ \hline
        -0.6 & 0.552067 & 0.549740 & 0.002327 & 0.995785 \\ \hline
        -0.5 & 0.339915 & 0.336128 & 0.003787 & 0.988859 \\ \hline
        -0.4 & 0.195987 & 0.192144 & 0.003843 & 0.980389 \\ \hline
        -0.3 & 0.101217 & 0.098325 & 0.002892 & 0.971428 \\ \hline
        -0.2 & 0.042215 & 0.040670 & 0.001545 & 0.963404 \\ \hline
        -0.1 & 0.010147 & 0.009719 & 0.000428 & 0.957814 \\ \hline
        0.0 & 0.000000 & -0.000000 & 0.000000 & 3.866301 \\ \hline
        0.1 & 0.010147 & 0.009719 & 0.000428 & 0.957814 \\ \hline
        0.2 & 0.042215 & 0.040670 & 0.001545 & 0.963404 \\ \hline
        0.3 & 0.101217 & 0.098325 & 0.002892 & 0.971428 \\ \hline
        0.4 & 0.195987 & 0.192144 & 0.003843 & 0.980389 \\ \hline
        0.5 & 0.339915 & 0.336128 & 0.003787 & 0.988859 \\ \hline
        0.6 & 0.552067 & 0.549740 & 0.002327 & 0.995785 \\ \hline
        0.7 & 0.858743 & 0.859280 & 0.000537 & 1.000625 \\ \hline
        0.8 & 1.295599 & 1.299898 & 0.004299 & 1.003318 \\ \hline
        0.9 & 1.910458 & 1.918426 & 0.007967 & 1.004170 \\ \hline
        1.0 & 2.767032 & 2.777260 & 0.010228 & 1.003696 \\ \hline
        1.1 & 3.949762 & 3.959557 & 0.009795 & 1.002480 \\ \hline
        1.2 & 5.570146 & 5.576042 & 0.005897 & 1.001059 \\ \hline
        1.3 & 7.774920 & 7.773759 & 0.001161 & 0.999851 \\ \hline
        1.4 & 10.756670 & 10.747136 & 0.009533 & 0.999114 \\ \hline
        1.5 & 14.767509 & 14.751773 & 0.015736 & 0.998934 \\ \hline
        1.6 & 20.136732 & 20.121373 & 0.015359 & 0.999237 \\ \hline
        1.7 & 27.293520 & 27.288318 & 0.005202 & 0.999809 \\ \hline
        1.8 & 36.796128 & 36.808375 & 0.012247 & 1.000333 \\ \hline
        1.9 & 49.369345 & 49.390083 & 0.020738 & 1.000420 \\ \hline
    \end{tabular}
    \caption{Вывод программы, представленный в табличном представлении для $f_1(x)$}
\end{table}


\begin{table}[!ht]
    \centering
    \begin{tabular}{|l|l|l|l|l|}
    \hline
        x & f (x) & P (x) + Узлы Чебышева & Абсолютная ошибка & Относительная ошибка \\ \hline
        -2.0 & -0.504968 & -0.467711 & 0.037257 & 0.926220 \\ \hline
        -1.9 & -0.505534 & -0.542968 & 0.037435 & 1.074050 \\ \hline
        -1.8 & -0.506060 & -0.530902 & 0.024841 & 1.049088 \\ \hline
        -1.7 & -0.506490 & -0.494587 & 0.011902 & 0.976500 \\ \hline
        -1.6 & -0.506731 & -0.466908 & 0.039823 & 0.921412 \\ \hline
        -1.5 & -0.506644 & -0.460188 & 0.046456 & 0.908306 \\ \hline
        -1.4 & -0.506017 & -0.473801 & 0.032216 & 0.936334 \\ \hline
        -1.3 & -0.504531 & -0.500012 & 0.004519 & 0.991044 \\ \hline
        -1.2 & -0.501706 & -0.528291 & 0.026585 & 1.052988 \\ \hline
        -1.1 & -0.496829 & -0.548337 & 0.051508 & 1.103674 \\ \hline
        -1.0 & -0.488853 & -0.552027 & 0.063174 & 1.129229 \\ \hline
        -0.9 & -0.476274 & -0.534472 & 0.058198 & 1.122195 \\ \hline
        -0.8 & -0.457001 & -0.494382 & 0.037381 & 1.081796 \\ \hline
        -0.7 & -0.428307 & -0.433889 & 0.005582 & 1.013033 \\ \hline
        -0.6 & -0.387018 & -0.357990 & 0.029028 & 0.924995 \\ \hline
        -0.5 & -0.330292 & -0.273740 & 0.056553 & 0.828779 \\ \hline
        -0.4 & -0.257406 & -0.189305 & 0.068100 & 0.735435 \\ \hline
        -0.3 & -0.172691 & -0.113001 & 0.059689 & 0.654357 \\ \hline
        -0.2 & -0.088411 & -0.052376 & 0.036036 & 0.592406 \\ \hline
        -0.1 & -0.024241 & -0.013425 & 0.010815 & 0.553832 \\ \hline
        0.0 & -0.000000 & -0.000000 & 0.000000 & 0.214459 \\ \hline
        0.1 & -0.024241 & -0.013425 & 0.010815 & 0.553832 \\ \hline
        0.2 & -0.088411 & -0.052376 & 0.036036 & 0.592406 \\ \hline
        0.3 & -0.172691 & -0.113001 & 0.059689 & 0.654357 \\ \hline
        0.4 & -0.257406 & -0.189305 & 0.068100 & 0.735435 \\ \hline
        0.5 & -0.330292 & -0.273740 & 0.056553 & 0.828779 \\ \hline
        0.6 & -0.387018 & -0.357990 & 0.029028 & 0.924995 \\ \hline
        0.7 & -0.428307 & -0.433889 & 0.005582 & 1.013033 \\ \hline
        0.8 & -0.457001 & -0.494382 & 0.037381 & 1.081796 \\ \hline
        0.9 & -0.476274 & -0.534472 & 0.058198 & 1.122195 \\ \hline
        1.0 & -0.488853 & -0.552027 & 0.063174 & 1.129229 \\ \hline
        1.1 & -0.496829 & -0.548337 & 0.051508 & 1.103674 \\ \hline
        1.2 & -0.501706 & -0.528291 & 0.026585 & 1.052988 \\ \hline
        1.3 & -0.504531 & -0.500012 & 0.004519 & 0.991044 \\ \hline
        1.4 & -0.506017 & -0.473801 & 0.032216 & 0.936334 \\ \hline
        1.5 & -0.506644 & -0.460188 & 0.046456 & 0.908306 \\ \hline
        1.6 & -0.506731 & -0.466908 & 0.039823 & 0.921412 \\ \hline
        1.7 & -0.506490 & -0.494587 & 0.011902 & 0.976500 \\ \hline
        1.8 & -0.506060 & -0.530902 & 0.024841 & 1.049088 \\ \hline
        1.9 & -0.505534 & -0.542968 & 0.037435 & 1.074050 \\ \hline
    \end{tabular}
    \caption{Вывод программы, представленный в табличном представлении для $f_2(x)$}
\end{table}
\tub Таблицу в формате \textbf{.xlsx} можно посомтреть в файле \textbf{table.xlsx}


\end{document}

